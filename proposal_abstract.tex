\documentclass[paper=a4,fontsize=12pt]{article}
\usepackage{geometry}
\usepackage{url}
\usepackage{enumitem}
\usepackage{hyperref}


\begin{document}
We propose a machine learning (ML) based algorithm to more efficiently identify the process of Vector Boson Fusion (VBF) within the ATLAS experiment. The ultimate goal is to observe the H->bbar process via the VBF production mechanism, complementary to the recent observation primarily via the more rare VH mechanism. The algorithm would use an understanding of VBF physics to guide its training; e.g., an approach that incorporates the topology of the VBF process to better label the many jets involved in the final state. We intend to bolster this algorithm with information from an in-development instrument called the High Granularity Timing Detector (HGTD). The HGTD provides high resolution timing information of vertices, allowing tracks and jets to be reconstructed much more reliably in the face of pileup. As SLAC is currently developing a prototype to the HGTD (ALTIROC1) it is ideally suited to host this project. The VBF algorithm can be designed using physical data from ALTIROC1, and the algorithm's development can act as a testing ground for ALTIROC1. As well, SLAC is host to world leading experts in the applications of ML to physics, and possess a vast suite of GPU clusters on which to perform algorithm training. Finally, our hope for this project would be that the techniques used in designing this algorithm could be applied to more complex analyses in the future, and that the feedback provided on ALTIROC1 will make for a more efficient final design of the HGTD.
\end{document}
