\documentclass[paper=a4,fontsize=12pt]{article}
							
\usepackage{geometry}
\usepackage{url}
\usepackage{enumitem}
\usepackage{hyperref}


\begin{document}

%VBF is important,
%but not easy
%Can be improved with HGTD, but that has its own issues
%all of this can be solved with Machine Learning
\section*{Background}
    Vector Boson Fusion is a process of great importance in high energy physics.
    It has something something important to do with understanding Higg's potential. %TODO
    Unfortunately, the process is a particularly challenging one to identify.
    Something something machine learning would fix all our problems %TODO 
    An additional tool that can help with the difficulties of vector boson fusion is the use of track timing information. Such information can be determined using a planned upgrade for ATLAS, the High Granularity Timing Detector (HGTD).






    %This is why VBF is hard
        %two jets produced by key event, but many more present in detector. Need to identify correct jet pair
    %Use directed machine learning technique to solve these problems
        %Instead of training nn to identify vbf generically, train it to identify key aspects of vbf separately 
\section*{Vector Boson Fusion and Machine Learning}
    The VBF process is a particularly challenging one to identify. The primary process needs to be identified by relating two pairs of jets. As these jets can overlap, the task of reconstructing and correctly identifying them is already difficult. The primary process is almost never isolated however, and is in most cases accompanied by a large amount of pileup, which makes the task of finding a VBF process much harder. While techiniques to identify VBF events exist, they are not as efficient as they could be, and cannot be easily extended to include new information about the process. However, such a complex pattern is exactly what machine learning techniques are ideal for. 


    %here's how it can help with VBF 
    %but might be tricky to use.
    %Best way to handle it may be ML,
    %hence why it's perfect for use with VBF NN. 
\section*{High Granularity Timing Detector}
    The High Granularity Timing Detector (HGTD) is a module planned for installation in ATLAS during the Phase-II upgrade. Its primary purpose is to provide precision (30 ps resolution) timing information of tracks tagged within ATLAS. Such timing information is invaluable in the context of VBF searches within ATLAS, where pileup results in several particles having vertices that partially or completely overlap each other. Currently, it is very difficult to associate seperate tracks with vertices that overlap in space. With the addition of the HGTD though, overlapping vertices will be distinguishable by their time-stamp, making them much easier to associate with the correct track. 
    
    One noteable issue with the HGTD is the prescence of material between itself and the tracker. This material causes scattering of particles entering the HGTD, and can lead to difficulty in determining which vertex in the HGTD should go to which vertices in the tracker. Once again, machine learning techniques may provide the best way to resolve this issue. Herein lies the advantage of combining research in VBF tagging with research on the HGTD. Using machine learning techniques to approach VBF tagging is by itself a worthwhile endevour. Using information from the HGTD to assist in VBF tagging would also be greatly valuable. However, since the HGTD could also benefit from a machine learning algorithm, it only seems appropriate to directly include information from the HGTD into the VBF machine learning algorithm. Doing so would make the best possible use of the HGTD's capabilities, and in doing so would significantly bolster the efficiency of the VBF tagging algorithm.


    
    
\section*{Here's why I should go to slac}
    SLAC is a center for machine learning research, with many experts in the field
    slac has powerful computing resources, which would be valuable in these studies
    HGTD and VBF synergize well as projects
    HGTD is great for VBF study, and VBF is a great testing ground for using HGTD.
    SLAC is ideal then b/c they have ALTIROC1, a prototype of HGTD.
    So I can do ML work on VBF, while getting hands on understanding of what hgtd can do (with actual testing) and how it works.
    At the same time, I can use my work on the vbf ml algorithm as feedback to steer development of hgtd.


\section*{looking forward}
    research done here is immediately helpful to vbf studies, but extends beyond them. 
    the ml techniques used here for two jets for vbf can be used for other physics events,
    and extended to more complex processes of more jets in the future.
    This can aid in the development of the hardware of the hgtd,
    and can act as a proof of concept of how hgtd output can be used via ml algorithms

    



\end{document}
