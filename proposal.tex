\documentclass[paper=a4,fontsize=12pt]{article}
							
\usepackage{geometry}
	\textheight=700px
\usepackage{url}
\usepackage{enumitem}
\usepackage{hyperref}


\begin{document}

%VBF is important,
%but not easy
%Can be improved with HGTD, but that has its own issues
%all of this can be solved with Machine Learning
\section*{Background}
    Vector Boson Fusion is a process of great importance in high energy physics.
    It has something something important to do with understanding Higg's potential. %TODO
    Unfortunately, the process is a particularly challenging one to identify.
    Something something machine learning would fix all our problems %TODO 
    An additional tool that can help with the difficulties of vector boson fusion is the use of track timing information. Such information can be determined using a planned upgrade for ATLAS, the High Granularity Timing Detector (HGTD).





%here's how ML techniques can deal with VBF stuff alone

\section*{VBF and Machine Learning}
    This is why VBF is hard
        two jets produced by key event, but many more present in detector. Need to identify correct jet pair
        final state radiation produced by event needs to be tagged, but can end up in quark jet
    Use directed machine learning technique to solve these problems
        Instead of training nn to identify vbf generically, train it to identify key aspects of vbf separately 


\section*{HGTD is neat}
    here's how it can help with VBF 
    but might be tricky to use.
    Best way to handle it may be ML,
    hence why it's perfect for use with VBF NN. 


\section*{Here's why I should go to slac}
    HGTD and VBF synergize well as projects
    HGTD is great for VBF study, and VBF is a great testing ground for using HGTD.
    SLAC is ideal then b/c they have ALTIROC1, a prototype of HGTD.
    So I can do ML work on VBF, while getting hands on understanding of what hgtd can do (with actual testing) and how it works.
    At the same time, I can use my work on the vbf ml algorithm as feedback to steer development of hgtd.


\section*{looking forward}
    research done here is immediately helpful to vbf studies, but extends beyond them. 
    the ml techniques used here for two jets for vbf can be used for other physics events,
    and extended to more complex processes of more jets in the future.
    This can aid in the development of the hardware of the hgtd,
    and can act as a proof of concept of how hgtd output can be used via ml algorithms

    



\end{document}
