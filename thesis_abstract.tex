\documentclass[paper=a4,fontsize=12pt]{article}
\usepackage{geometry}
\usepackage{url}
\usepackage{enumitem}
\usepackage{hyperref}


\begin{document}
The interaction of the Higgs boson with the bottom quark is predicted by the Standard Model (SM) of particle physics to be large, leading to a branching fraction of H->bb of about 60\%. While the process has been observed for the first time in 2018, independent study is required to improve our knowledge while also continuing to test the SM. To this end an analysis of ATLAS Run 2 data at 140 inverse femtobarns is conducted in order to better probe the Higgs coupling to quarks. The Higgs is studied using the process of vector boson fusion into a pair of quark jets, which accounts for the dominant decay mode of the Higgs. Traditionally, the vector boson fusion production mechanism has been a difficult one to study, as its primary signature is a number of jets. In this analysis however, we use an advanced machine learning algorithm technique to identify VBF processes. The algorithm is informed by a high-level theoretical understanding of the VBF process, allowing for much more efficient training and selection. The usefulness of this ML-based approach is also studied in related contexts, such as the production of the Higgs with other objects such as single high-momentum jets or additional Higgs bosons that also decay hadronically.
\end{document}
